\documentclass{sig-alternate}
\usepackage[utf8]{inputenc}
\usepackage{graphicx}
\usepackage{subfigure}
\usepackage{color}
\usepackage{multicol}
\usepackage[hyphens]{url}
\usepackage{xspace}
\usepackage{microtype}
%\usepackage[ruled,linesnumbered]{algorithm2e}
\usepackage{multirow, listings, longtable, afterpage, pdflscape,
  tabularx, calc}
  
\newcommand{\ap}{APEG\hspace{2pt}}  

\def\sharedaffiliation{
\end{tabular}
\begin{tabular}{c}}

\pagenumbering{arabic}

\begin{document}

%opening
\title{Towards Automatic Exploit Generation for Android C Libraries}
\author{
\alignauthor
Bhargava Shastry, Lee SeoungKyou\\
\sharedaffiliation
      \affaddr{Rice University}\\
      \affaddr{Houston, U.S.A.}\\
}

\maketitle

\begin{abstract}
Managing the security of complex systems poses a great challenge. Incremental software patches are by far the most common means of plugging security holes: be it at the level of the operating system kernel, the middleware or the applications.
However, the practice of system hardening via patch deployment assumes that the concerned patches are deployed uniformly across systems running a piece of software. This is far from reality: software vendors often stagger the deployment of patches for various reasons. 
This project is a step in the direction of building a static analysis framework that is able to analyse differences in code across multiple versions of a component in a large software system and provide as output, inputs to these components that might compromise security of the system. We choose the Android mobile platform as a case study where our framework could be put to test.
\end{abstract}

\section{Introduction} \label{sec:intro}
% Managing the security of complex systems poses a great challenge. Incremental software patches are by far the most common means of plugging security holes: be it at the level of the operating system kernel, the middleware or the applications.

Contrary to the view that security patches enhance the security of a software system, recent work~\cite{apeg08}---henceforth refered to as \ap---shows how the former can actually aid attackers in exploiting the very same vulnerabilities that were patched by exploiting the vulnerability on the unpatched version of the program. Note that this is nontrivial: a security patch need not reveal the vulnerability in a straightforward way; on average, it would take a seasoned hacker a few days of work at the very least to write an exploit to undermine an unpatched program. One would need to analyse the target program to list program variables that are effected by the patch and where they are used and ultimately craft an input that would attack the unpatched program.

A good example of staggered patch deployment is the Android mobile platform, where, due to fragmented variants of the platform (hardware and software), different phones seem to be running different versions of the Android operating system. The extent of fragmentation has been reported by OpenSignal~\cite{opensignal}. It has to be noted that not only is the device (hardware) fragmented, but so is the entire software stack being used by a given device; the API-level (Android version) and the underlying Linux kernel version. Android's open nature has led to a proliferation of devices, each with subtly different hardware, running different variants of the software stack, not to talk about custom ROMs that are also available.

How is automatic patch-based exploit generation relavant to Android? The staggered nature of patch deployment is more of a problem for an open platform like Android than other closed mobile platforms like Apple's iOS. Android's customisability is a double-edged sword---while users get to choose from multiple varied devices, app developers need to cater their offerings to each of these devices. More importantly, from a security point-of-view, vendors need to ensure that relevant security patches are deployed uniformly across all devices from time to time. The reality is far from what one desires: \TODO{cite reports here}

\section{Problem Statement} \label{prob_st}
Formally, \ap proposes techniques for the automatic patch-based exploit generation problem: given a program $P$, and a patched version of the same program, $P'$, automatically generate an exploit for the potentially unknown vulnerability in $P$ but patched in $P'$. Furthermore, the work targets input validation vulnerabilities only: vulnerabilities arising out of invalid inputs to the program. Hence, the attack can be modelled as a constraint satisfaction problem, with the constraint that an input fail the input sanitisation checks added in $P'$ with an assumption that there is an exploit corresponding to a given input validation vulnerability. \ap argues that given that the delivery of patches is staggered over long time periods, there is a realistic time window in which unpatched systems are vulnerable to attack.

In this project, we seek to emulate the work done in~\cite{apeg08} in a different setting. More precisely, our setting is different from theirs in the following ways:
\begin{enumerate}
 \item \ap works on closed-source binaries (executables), while we only consider open-source software. This lets us focus on the problem of automatic exploit generation and not worry about reverse engineering a patch. Specifically, our target of attack would be userspace libraries (written in C language) in the Android software stack e.g., \textit{bionic} libc, dalvik runtime libraries etc.
 \item \ap proposes techniques based on dynamic program analysis only, static program analysis only and a mix of both, while we focus on static analysis only.
\end{enumerate}

\subsection{Methodology}\label{meth}
The project could be logically divided into the following phases:
\begin{enumerate}
 \item Identify what constitutes a security patch for the attack target. This would need to be manually by looking up the commit history (log) of the target software.\footnote[1]{Android maintains a git repository for each component in its software stack. Commits are always tagged with textual comments about changes done.}
 \item From the security patches thus identified, filter out those that do NOT address an input validation vulnerability e.g., focus on sanitisation checks added on program variables. The patches that remain are those that will constitute our exploitable set.
 \item Using techniques from static analysis, identify which part of the unpatched software has an input validation vulnerability. This phase is going to be challenging becuase it is nontrivial to search for a variable sanitised in $P'$ that is used as an input in $P$. We would be relying upon the CIL infrastructure~\cite{cilpaper} for obtaining an intermediate representation (Abstract Syntax Tree) of the target software that is amenable to static analysis.
 \item Express the problem of finding an exploitable input as a constraint satisfaction problem and rely upon Satisfiability Modulo Theories (SMT) solvers to obtain a solution to the problem.
\end{enumerate}

\subsection{Deliverables}\label{del}
Building a framework for statically analysing the target software for possible input validation vulnerabilities would be a deliverable. However, the most important deliverable of this project would be (hopefully) a few hitherto unknown input validation vulnerabilities on a small portion (native libraries) of Android's software stack.

\section{Related Work} \label{rel_work}

\section{Framework} \label{framework}

\section{Deliverables} \label{del}
One of the main goals of this project was to develop a framework that is capable of performing static analysis on C code. We believe our project takes this goal forward. We have a proof of concept static analysis framework that is capable of inferring simple relationships between program variables. The present framework has several limitations, namely, (1) It works for a limited subset of C language instructions and statements; prominently, pointers are not handled; (2) It does not work well for loops in general; we intend to make static analysis in the presence of loops more precise; (3) Finally, our framework is presently modeled on a simplistic constraint language consisting of boolean operators and the less than or equal to operator, $\le$; for instance, at each point in a C program, our framework keeps a track of constraints in a first-order logical formula consisting of logical operators $\neg$, $\land$, $\lor$, and the $\le$ arithmetic operator. Furthermore, we confine our analyis to linear arithmetic expressions (combination of $+$ and $\times$ operators on program variables and constants)

\section{Conclusion and Future Work} \label{conclusion}

\bibliographystyle{abbrv}
\bibliography{comp527}

\end{document}
