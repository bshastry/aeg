\documentclass{sig-alternate}
\usepackage[utf8]{inputenc}
\usepackage{graphicx}
\usepackage{subfigure}
\usepackage{color}
\usepackage{multicol}
\usepackage[hyphens]{url}
\usepackage{xspace}
\usepackage{microtype}
\usepackage{algorithm}
\usepackage{algorithmic}
%\usepackage[ruled,linesnumbered]{algorithm2e}
\usepackage{multirow, listings, longtable, afterpage, pdflscape,
  tabularx, calc}
  
\newcommand{\ap}{APEG\hspace{2pt}}
\newcommand{\TODO}[1]{\textcolor{red}{\textbf{TODO: #1}}}

\def\sharedaffiliation{
\end{tabular}
\begin{tabular}{c}}

\pagenumbering{arabic}

\begin{document}

%opening
\title{Towards Automatic Exploit Generation for Android C Libraries}
\author{
\alignauthor
Bhargava Shastry, Lee SeoungKyou\\
\sharedaffiliation
      \affaddr{Rice University}\\
      \affaddr{Houston, USA}\\
}

\maketitle

\begin{abstract}
Managing the security of complex systems poses a great challenge. Incremental software patches are by far the most common means of plugging security holes: be it at the level of the operating system kernel, the middleware or the applications.
However, the practice of system hardening via patch deployment assumes that the concerned patches are deployed uniformly across systems running a piece of software. This is far from reality: software vendors often stagger the deployment of patches for various reasons. 
This project is a step in the direction of building a static analysis framework that is able to analyse differences in code across multiple versions of a component in a large software system and provide as output, inputs to these components that might compromise security of the system. We choose the Android mobile platform as a case study where our framework could be put to test.
\end{abstract}

\section{Introduction} \label{sec:intro}
Managing the security of complex systems poses a great challenge. Incremental software patches are by far the most common means of plugging security holes: be it at the level of the operating system kernel, the middleware or the applications.

Contrary to the view that security patches enhance the security of the software system, a recent work~\cite{apeg08}---henceforth refered to as \ap---shows that the former can actually help attackers in exploiting the very same vulnerabilities that were patched by exploiting the vulnerability on the unpatched system. Note that this is nontrivial: a security patch need not reveal the vulnerability in a straightforward way. One would need to analyse the target program to list program variables that are effected by the patch and where they are used. This entails significant work.


A good example of staggered patch deployment is the Android mobile platform, where, due to fragmented variants of the platform (hardware and software), different phones seem to be running different versions of the Android operating system.

\section{Problem Statement} \label{prob_st}
Formally, \ap proposes techniques for the automatic patch-based exploit generation problem: given a program $P$, and a patched version of the same program, $P'$, automatically generate an exploit for the potentially unknown vulnerability in $P$ but patched in $P'$. Furthermore, the work targets input validation vulnerabilities only: vulnerabilities arising out of invalid inputs to the program. Hence, the attack can be modelled as a constraint satisfaction problem, with the constraint that an input fail the input sanitisation checks added in $P'$ with an assumption that there is an exploit corresponding to a given input validation vulnerability. \ap argues that given that the delivery of patches is staggered over long time periods, there is a realistic time window in which unpatched systems are vulnerable to attack.

In this project, we seek to emulate the work done in~\cite{apeg08} in a different setting. More precisely, our setting is different from theirs in the following ways:
\begin{enumerate}
 \item \ap works on closed-source binaries (executables), while we only consider open-source software. This lets us focus on the problem of automatic exploit generation and not worry about reverse engineering a patch. Specifically, our target of attack would be userspace libraries (written in C language) in the Android software stack e.g., \textit{bionic} libc, dalvik runtime libraries etc.
 \item \ap proposes techniques based on dynamic program analysis only, static program analysis only and a mix of both, while we focus on static analysis only.
\end{enumerate}

\subsection{Methodology}\label{meth}
The project could be logically divided into the following phases:
\begin{enumerate}
 \item Identify what constitutes a security patch for the attack target. This would need to be manually by looking up the commit history (log) of the target software.\footnote[1]{Android maintains a git repository for each component in its software stack. Commits are always tagged with textual comments about changes done.}
 \item From the security patches thus identified, filter out those that do NOT address an input validation vulnerability e.g., focus on sanitisation checks added on program variables. The patches that remain are those that will constitute our exploitable set.
 \item Using techniques from static analysis, identify which part of the unpatched software has an input validation vulnerability. This phase is going to be challenging becuase it is nontrivial to search for a variable sanitised in $P'$ that is used as an input in $P$. We would be relying upon the CIL infrastructure~\cite{cilpaper} for obtaining an intermediate representation (Abstract Syntax Tree) of the target software that is amenable to static analysis.
 \item Express the problem of finding an exploitable input as a constraint satisfaction problem and rely upon Satisfiability Modulo Theories (SMT) solvers to obtain a solution to the problem.
\end{enumerate}

\subsection{Deliverables}\label{del}
Building a framework for statically analysing the target software for possible input validation vulnerabilities would be a deliverable. However, the most important deliverable of this project would be (hopefully) a few hitherto unknown input validation vulnerabilities on a small portion (native libraries) of Android's software stack.

\section{Related Work} \label{rel_work}

\section{Background} \label{background}
\TODO{Talk of the following things: (1) Theory of program verification; program annotations for verification, pre-condition, post-condition; (2) Control-flow graph and its computation; (3) Approach employed in this project, perhaps a running example.}

We will use the running example of a program (Algorithm~\ref{alg1}) that computes the $n^{th}$ power of a given integer to introduce the reader to program verification terminology. 

\begin{algorithm}                      % enter the algorithm environment
\caption{Calculate $y = x^n$}          % give the algorithm a caption
\label{alg1}                           % and a label for \ref{} commands later in the document
\begin{algorithmic}[1]                    % enter the algorithmic environment
    \REQUIRE $n \geq 0 \vee x \neq 0$
    \ENSURE $y = x^n$
    \STATE $y \Leftarrow 1$
    \IF{$n < 0$}
        \STATE $X \Leftarrow 1 / x$
        \STATE $N \Leftarrow -n$
    \ELSE
        \STATE $X \Leftarrow x$
        \STATE $N \Leftarrow n$
    \ENDIF
    \WHILE{$N \neq 0$}
        \IF{$N$ is even}
            \STATE $X \Leftarrow X \times X$
            \STATE $N \Leftarrow N / 2$
        \ELSE[$N$ is odd]
            \STATE $y \Leftarrow y \times X$
            \STATE $N \Leftarrow N - 1$
        \ENDIF
    \ENDWHILE
\end{algorithmic}
\end{algorithm}

Let Algorithm~\ref{alg1} refer to a program P, consisting of statements $S_{1}, S{2}, ..., S{17}$. The pre-condition, $pre$, of program P formalises conditions that are required to be satisfied by the program prior to its execution; in the present example, it is a function of $x$ and $n$. The post-condition of the program, $post$, formalises conditions that are required to be satisfied by the program after its execution. Note that $pre$ and $post$ are both expressions in first order logic i.e., a boolean predicate.

Program statements $S_{1}, ... , S_{17}$ may contain variables ($x, y\in V$), arithmetic and boolean expressions ($E$), and be composed of assignment statements ($V:=E$) , if-then-else statements, and while loops.

The triple $pre$\{$S$\}$post$ is called a Hoare triple named after C.A.R. Hoare, a pioneer in the field of program verification. The Hoare triple is to be read as follows: If the pre-condition is satisfied by a program P consisting of statements $S_{i} \in S ,	i=1,...,n$, then, on execution of the statements in $S$, the post-condition will necessarily be satisfied should the program be correct and assuming that it terminates.

\subsection{Weakest preconditions and strongest post conditions}
We borrow introductory material and definitions of the weakest pre-condition and strongest post-condition in this section from~\cite{Gordon10}---a good survey of verification methods used today.

Consider two boolean predicates $A$ and $B$. If the relation $A\implies B$ is true, then $A$ is said to be stronger than $B$ (since A implies B); conversely $B$ is said to be weaker than $A$.

The strongest post-condition of a program $P$ with a pre-condition $pre$ is denoted by $Sp\{P\}\ pre$. The strongest post-condition is the predicate obtained by transforming the pre-condition according to the interpretation of program statements in $P$. The strongest post-condition is termed so because, given a predicate $post$ that is true of any state obtained by interpreting the program $P$ from an initial state where $pre$ holds, $Sp\{P\}\ pre\implies post$.

The weakest pre-condition of a program with a post-condition $post$ is denoted by $Wp\{P\}\ post$. The weakest pre-condition is the predicate obtained by transforming the post-condition according to the interpretation of program statements in $P$ in reverse order. The weakest pre-condition is weaker than the $pre$ in that $pre\implies Wp\{P\}\ post$.

A relation between the Hoare triple $pre$\{$S$\}$post$ for a given program $P$ and the weakest pre-condition $Wp$ and the strongest post-condition $Sp$ is that the Hoare triple holds if and only if $(Sp\ S\ pre)\implies post$ or if and only if $pre\implies (Wp\ S\ post)$.


\section{Framework} \label{framework}
\TODO{Talk about what we have built: a framework that is capable of statically analysing C code. How far are we from what we set out to do? Our framework can currently scan simple C code and infer certain properties (abstractions about program variables) about it. This has to be applied to the problem of exploit generation; specifically generating weakest pre-condition, given a program and a patch.
Tools we use:
Ocaml: functional language that is type safe and models effects.
CIL: Compiler front-end for C programs. Good for static analysis.
CIL-template: Sample programs for CIL. Build scripts reused.
}
Our framework for static analysis is modeled for a subset of the C language: it supports simple linear arithmetic---the use of $+$ and $\times$ operator on program variables and constants. 



\section{Deliverables} \label{del}
One of the main goals of this project was to develop a framework that is capable of performing static analysis on C code. We believe our project takes this goal forward. We have a proof of concept static analysis framework that is capable of inferring simple relationships between program variables. The present framework has several limitations, namely, (1) It works for a limited subset of C language instructions and statements; prominently, pointers are not handled; (2) It does not work well for loops in general; we intend to make static analysis in the presence of loops more precise; (3) Finally, our framework is presently modeled on a simplistic constraint language consisting of boolean operators and the less than or equal to operator, $\le$; for instance, at each point in a C program, our framework keeps a track of constraints in a first-order logical formula consisting of logical operators $\neg$, $\land$, $\lor$, and the $\le$ arithmetic operator. Furthermore, we confine our analyis to linear arithmetic expressions (combination of $+$ and $\times$ operators on program variables and constants)

\section{Conclusion and Future Work} \label{conclusion}

\bibliographystyle{abbrv}
\bibliography{comp527}

\end{document}
