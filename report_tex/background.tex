\section{Background} \label{background}
\TODO{Talk of the following things: (1) Theory of program verification; program annotations for verification, pre-condition, post-condition; (2) Control-flow graph and its computation; (3) Approach employed in this project, perhaps a running example.}

We will use the running example of a program (Algorithm~\ref{alg1}) that computes the $n^{th}$ power of a given integer to introduce the reader to program verification terminology. 

\begin{algorithm}                      % enter the algorithm environment
\caption{Calculate $y = x^n$}          % give the algorithm a caption
\label{alg1}                           % and a label for \ref{} commands later in the document
\begin{algorithmic}[1]                    % enter the algorithmic environment
    \REQUIRE $n \geq 0 \vee x \neq 0$
    \ENSURE $y = x^n$
    \STATE $y \Leftarrow 1$
    \IF{$n < 0$}
        \STATE $X \Leftarrow 1 / x$
        \STATE $N \Leftarrow -n$
    \ELSE
        \STATE $X \Leftarrow x$
        \STATE $N \Leftarrow n$
    \ENDIF
    \WHILE{$N \neq 0$}
        \IF{$N$ is even}
            \STATE $X \Leftarrow X \times X$
            \STATE $N \Leftarrow N / 2$
        \ELSE[$N$ is odd]
            \STATE $y \Leftarrow y \times X$
            \STATE $N \Leftarrow N - 1$
        \ENDIF
    \ENDWHILE
\end{algorithmic}
\end{algorithm}

Let Algorithm~\ref{alg1} refer to a program P, consisting of statements $S_{1}, S{2}, ..., S{17}$. The pre-condition, $pre$, of program P formalises conditions that are required to be satisfied by the program prior to its execution; in the present example, it is a function of $x$ and $n$. The post-condition of the program, $post$, formalises conditions that are required to be satisfied by the program after its execution. Note that $pre$ and $post$ are both expressions in first order logic i.e., a boolean predicate.

Program statements $S_{1}, ... , S_{17}$ may contain variables ($x, y\in V$), arithmetic and boolean expressions ($E$), and be composed of assignment statements ($V:=E$) , if-then-else statements, and while loops.

The triple $pre$\{$S$\}$post$ is called a Hoare triple named after C.A.R. Hoare, a pioneer in the field of program verification. The Hoare triple is to be read as follows: If the pre-condition is satisfied by a program P consisting of statements $S_{i} \in S ,	i=1,...,n$, then, on execution of the statements in $S$, the post-condition will necessarily be satisfied should the program be correct and assuming that it terminates.

\subsection{Weakest preconditions and strongest post conditions}
We borrow introductory material and definitions of the weakest pre-condition and strongest post-condition in this section from~\cite{Gordon10}---a good survey of verification methods used today.

Consider two boolean predicates $A$ and $B$. If the relation $A\implies B$ is true, then $A$ is said to be stronger than $B$ (since A implies B); conversely $B$ is said to be weaker than $A$.

The strongest post-condition of a program $P$ with a pre-condition $pre$ is denoted by $Sp\{P\}\ pre$. The strongest post-condition is the predicate obtained by transforming the pre-condition according to the interpretation of program statements in $P$. The strongest post-condition is termed so because, given a predicate $post$ that is true of any state obtained by interpreting the program $P$ from an initial state where $pre$ holds, $Sp\{P\}\ pre\implies post$.

The weakest pre-condition of a program with a post-condition $post$ is denoted by $Wp\{P\}\ post$. The weakest pre-condition is the predicate obtained by transforming the post-condition according to the interpretation of program statements in $P$ in reverse order. The weakest pre-condition is weaker than the $pre$ in that $pre\implies Wp\{P\}\ post$.

A relation between the Hoare triple $pre$\{$S$\}$post$ for a given program $P$ and the weakest pre-condition $Wp$ and the strongest post-condition $Sp$ is that the Hoare triple holds if and only if $(Sp\ S\ pre)\implies post$ or if and only if $pre\implies (Wp\ S\ post)$.
