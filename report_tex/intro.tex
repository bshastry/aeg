\section{Introduction} \label{sec:intro}
% Managing the security of complex systems poses a great challenge. Incremental software patches are by far the most common means of plugging security holes: be it at the level of the operating system kernel, the middleware or the applications.

Contrary to the view that security patches enhance the security of a software system, recent work~\cite{apeg08}---henceforth refered to as \ap---shows how the former can actually aid attackers in exploiting the very same vulnerabilities that were patched by exploiting the vulnerability on the unpatched version of the program. Note that this is nontrivial: a security patch need not reveal the vulnerability in a straightforward way; on average, it would take a seasoned hacker a few days of work at the very least to write an exploit to undermine an unpatched program. One would need to analyse the target program to list program variables that are effected by the patch and where they are used and ultimately craft an input that would attack the unpatched program.

A good example of staggered patch deployment is the Android mobile platform, where, due to fragmented variants of the platform (hardware and software), different phones seem to be running different versions of the Android operating system. The extent of fragmentation has been reported by OpenSignal~\cite{opensignal}. It has to be noted that not only is the device (hardware) fragmented, but so is the entire software stack being used by a given device; the API-level (Android version) and the underlying Linux kernel version. Android's open nature has led to a proliferation of devices, each with subtly different hardware, running different variants of the software stack, not to talk about custom ROMs that are also available.

How is automatic patch-based exploit generation relavant to Android? The staggered nature of patch deployment is more of a problem for an open platform like Android than other closed mobile platforms like Apple's iOS. Android's customisability is a double-edged sword---while users get to choose from multiple varied devices, app developers need to cater their offerings to each of these devices. More importantly, from a security point-of-view, vendors need to ensure that relevant security patches are deployed uniformly across all devices from time to time. The reality is far from what one desires: \TODO{cite reports here}